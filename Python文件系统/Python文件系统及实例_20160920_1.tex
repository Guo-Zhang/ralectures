%-*-coding:utf-8-*-
\documentclass[13pt]{beamer}

\mode<presentation>
{
\usetheme{Warsaw}

\setbeamercovered{transparent}
}

\usepackage[UTF8]{ctex} % 支持中文
%\usepackage[encoding=UTF8]{zhnumber} % 中文日期
\usepackage[english]{babel} %选择语言
%\usepackage[latin1]{inputenc}

% font definitions, try \usepackage{ae} instead of the following
% three lines if you don't like this look
\usepackage{mathptmx}
\usepackage[scaled=.90]{helvet}
\usepackage{courier}

\usepackage[T1]{fontenc}

\usepackage[utf8]{inputenc}

% Default fixed font does not support bold face
\DeclareFixedFont{\ttb}{T1}{txtt}{bx}{n}{12} % for bold
\DeclareFixedFont{\ttm}{T1}{txtt}{m}{n}{12}  % for normal

% Custom colors
\usepackage{color}
\definecolor{deepblue}{rgb}{0,0,0.5}
\definecolor{deepred}{rgb}{0.6,0,0}
\definecolor{deepgreen}{rgb}{0,0.5,0}

\usepackage{listings}

% Python style for highlighting
\newcommand\pythonstyle{\lstset{
language=Python,
basicstyle=\ttm,
otherkeywords={self},             % Add keywords here
keywordstyle=\ttb\color{deepblue},
emph={MyClass,__init__},          % Custom highlighting
emphstyle=\ttb\color{deepred},    % Custom highlighting style
stringstyle=\color{deepgreen},
frame=tb,                         % Any extra options here
showstringspaces=false            % 
}}


% Python environment
\lstnewenvironment{python}[1][]
{
\pythonstyle
\lstset{#1}
}
{}

% Python for external files
\newcommand\pythonexternal[2][]{{
\pythonstyle
\lstinputlisting[#1]{#2}}}

% Python for inline
\newcommand\pythoninline[1]{{\pythonstyle\lstinline!#1!}}


% If you have a file called "university-logo-filename.xxx", where xxx
% is a graphic format that can be processed by latex or pdflatex,
% resp., then you can add a logo as follows:

% \pgfdeclareimage[height=0.5cm]{university-logo}{university-logo-filename}
% \logo{\pgfuseimage{university-logo}}



% Delete this, if you do not want the table of contents to pop up at
% the beginning of each subsection:
\AtBeginSubsection[]
{
\begin{frame}<beamer>
\frametitle{目录}
\tableofcontents[currentsection,currentsubsection]
\end{frame}
}

% If you wish to uncover everything in a step-wise fashion, uncomment
% the following command:

%\beamerdefaultoverlayspecification{<+->}

\begin{document}


\title{Python文件操作}

%\subtitle{}

% - Use the \inst{?} command only if the authors have different
%   affiliation.
%\author{F.~Author\inst{1} \and S.~Another\inst{2}}
\author{张果}

% - Use the \inst command only if there are several affiliations.
% - Keep it simple, no one is interested in your street address.
\institute[Universities of]
{
厦门大学\quad 王亚南经济研究院}

%\ctexset{today=small} %中文日期
\renewcommand{\today}{\number\year 年 \number\month 月 \number\day 日}
\date{\today}


% This is only inserted into the PDF information catalog. Can be left
% out.
\subject{Talks}




\begin{frame}
\titlepage
\end{frame}

\begin{frame}
\frametitle{目录}
\tableofcontents
% You might wish to add the option [pausesections]
\end{frame}


\begin{frame}
\frametitle{相关模块}
\begin{itemize}
  \item<1-> os:文件系统的访问
  \item<2-> os.path:管理和操作路径名
\end{itemize}

\end{frame}

\section{os模块}

\begin{frame}
\frametitle{文件处理}
\begin{itemize}
  \item<1-> os.remove(fname)  \# 删除文件
  \item<2-> os.rename(oldname,newname)  \# 重命名/移动文件
  \item<3-> os.walk(path) \# 生成一个目录树下所有文件
\end{itemize}
\end{frame}

\begin{frame}
\frametitle{目录、文件夹}
\begin{itemize}
  \item<1-> os.chdir(path)  \# 改变当前工作目录
  \item<2-> os.listdir(path)  \# 列出指定目录的文件名
  \item<3-> os.getcwd() \# 返回当前工作目录
  \item<4-> os.mkdir(path) \# 创建目录
\end{itemize}
\end{frame}

\section{os.path模块}
\begin{frame}
\frametitle{分隔}
\begin{itemize}
  \item<1-> os.path.basename(fname) \# 去掉目录路径,返回文件名
  \item<2-> os.path.dirname(fname)  \# 去掉文件名,返回目录路径
  \item<3-> os.path.join(dirname,basename) \# 组合目录名和文件名
  \item<4-> os.path.split(fname) \# 拆分目录名和文件名
\end{itemize}
\end{frame}

\begin{frame}
\frametitle{信息}
\begin{itemize}
  \item<1-> os.path.exists(path or fname) \# 指定路径(文件or目录)
  \item<2-> os.path.isfile(fname)  指定路径是否存在且为一个文件
  \item<3-> os.path.isdir(dirname)  指定路径是否存在且为一个目录
\end{itemize}
\end{frame}

\section{一些实例}
\begin{frame}
\frametitle{批量集中文件}
\begin{itemize}
  \item<1-> 目标:把一个文件树下全部文件集中到首级文件夹下。
  \item<2-> 思路:
  \pause
  \begin{itemize}
    \item 遍历文件树:os.walk
    \pause
    \item 把文件树下每一个文件移动到首级文件夹:for;os.rename
   \end{itemize}
  \item<3-> 实例:sortdata.py
  \item<4-> 拓展:
  \pause
  \begin{itemize}
    \item 按拓展名分类文件:if;in
    \pause
    \item 移动到特定文件夹
  \end{itemize}
\end{itemize}
\end{frame}

\begin{frame}
\frametitle{批量分类文件}
  \begin{itemize}
  \item<1-> 目标:把一个文件夹中的文件分成若干500个文件的文件夹。
  \item<2-> 思路:
  \pause
  \begin{itemize}
    \item 遍历文件夹下所有文件:for;os.listdir
    \pause
    \item 创建新文件夹:if;os.path.exists;os.mkdir
    \pause
    \item 把文件移动到新文件夹下:os.path.join;os.rename
  \end{itemize}
  \item<3-> 实例:divpdf.py
  \item<4-> 拓展:
  \pause
  \begin{itemize}
    \item 按拓展名分类文件:if;in
    \pause
    \item 移动到特定文件夹:if;in
  \end{itemize}
  \end{itemize}
\end{frame}


\begin{frame}
\frametitle{批量筛选不需要的PDF}
  \begin{itemize}
  \item<1-> 目标:把英文、已取消、补充公告的财报PDF取出。
  \item<2-> 思路:
  \pause
  \begin{itemize}
    \item 创建新文件夹:if;os.path.exists;os.mkdir
    \pause
    \item 遍历文件夹下所有文件:for;os.listdir
    \pause
    \item 判断文件是不是需要筛选出来:if;in
    \pause
    \item 把文件移动到新文件夹下:os.path.join;os.rename
   \end{itemize}
  \item<3-> 实例:selectPassPDF.py
  \end{itemize}
\end{frame}

\begin{frame}
\frametitle{批量筛选已处理PDF}
\begin{itemize}
  \item<1-> 目标:把已处理PDF(即有对应txt文件的PDF)取出。
  \item<2-> 思路:
  \pause
  \begin{itemize}
    \item 创建新文件夹:if;os.path.exists;os.mkdir
    \pause
    \item 遍历TXT文件夹下所有文件:for;os.listdir
    \pause
    \item 判断对应PDF文件是否已经被筛选:str.replace;if;in
    \pause
    \item 把PDF文件移动到新文件夹下:os.path.join;os.rename
   \end{itemize}
  \item<3-> 实例:selectPassPDF.py
\end{itemize}
\end{frame}

\begin{frame}
\frametitle{批量提取文件内容}
\begin{itemize}
  \item<1-> 目标:批量提取原始文件名中的URL。
  \item<2-> 思路:
  \pause
  \begin{itemize}
    \item 打开文件:open
    \pause
    \item 读入文件内容:file.readlines()
    \pause
    \item 从每行提取URL:str.split
    \pause
    \item 写入新文件:open; file.write
   \end{itemize}
  \item<3-> 实例:urlsort.py

\end{itemize}
\end{frame}

\begin{frame}
\frametitle{批量重命名PDF}
\begin{itemize}
  \item<1-> 目标:批量重命名用迅雷下载下来的PDF。
  \item<2-> 思路:
  \begin{itemize}
    \pause
    \item 遍历文件夹下所有文件:for;os.listdir
    \pause
    \item 生成新旧文件名
    
    \begin{itemize}
      \pause
      \item 读取文件信息
      \pause
      \item 切割每行字符串
      \pause
      \item 提取新旧文件名
    \end{itemize}
    \pause
    \item 把旧文件名重命名成新文件名:os.path.join;os.rename
   \end{itemize}
  \item<3-> 实例:rename.py
\end{itemize}
\end{frame}

\begin{frame}
\frametitle{更多}
\begin{itemize}
\item<1-> 从网站上批量下载PDF:爬虫
\item<2-> 检查TXT文件编码:文件编码
\item<3-> 统计TXT文件:正则表达式
\item ……
\end{itemize}
\end{frame}

\begin{frame}
\frametitle{作业}
\begin{itemize}
  \item 实例3-4 基础
  \item 实例5-6 选做
\end{itemize}
\end{frame}

% \begin{frame}
% \frametitle{}

% You can create overlays
% \begin{itemize}
%   \item using the \texttt{pause} command:
%   \begin{itemize}
%     \item First item.
%     \pause
%     \item Second item.
%   \end{itemize}
%   \item using overlay specifications:
%   \begin{itemize}
%     \item<3-> First item.
%     \item<4-> Second item.
%   \end{itemize}
%   \item using the general \texttt{uncover} command:
%   \begin{itemize}
%     \uncover<5->{\item First item.}
%     \uncover<6->{\item Second item.}
%   \end{itemize}
% \end{itemize}
% \end{frame}
% 
% \section*{Summary}
% 
% \begin{frame}
% \frametitle<presentation>{Summary}
% 
% \begin{itemize}
%   \item The \alert{first main message} of your talk in one or two lines.
% \end{itemize}
% 
% % The following outlook is optional.
% \vskip0pt plus.5fill
% \begin{itemize}
%   \item Outlook
%   \begin{itemize}
%     \item Something you haven't solved.
%     \item Something else you haven't solved.
%   \end{itemize}
% \end{itemize}
% \end{frame}

\end{document}
