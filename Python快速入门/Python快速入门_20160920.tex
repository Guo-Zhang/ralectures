%-*-coding:utf-8-*-
\documentclass[13pt]{beamer}

\mode<presentation>
{
\usetheme{Warsaw}

\setbeamercovered{transparent}
}

\usepackage[UTF8]{ctex} % 支持中文
%\usepackage[encoding=UTF8]{zhnumber} % 中文日期
\usepackage[english]{babel} %选择语言
%\usepackage[latin1]{inputenc}

% font definitions, try \usepackage{ae} instead of the following
% three lines if you don't like this look
\usepackage{mathptmx}
\usepackage[scaled=.90]{helvet}
\usepackage{courier}

\usepackage[T1]{fontenc}

\usepackage[utf8]{inputenc}

% Default fixed font does not support bold face
\DeclareFixedFont{\ttb}{T1}{txtt}{bx}{n}{12} % for bold
\DeclareFixedFont{\ttm}{T1}{txtt}{m}{n}{12}  % for normal

% Custom colors
\usepackage{color}
\definecolor{deepblue}{rgb}{0,0,0.5}
\definecolor{deepred}{rgb}{0.6,0,0}
\definecolor{deepgreen}{rgb}{0,0.5,0}

\usepackage{listings}

% Python style for highlighting
\newcommand\pythonstyle{\lstset{
language=Python,
basicstyle=\ttm,
otherkeywords={self},             % Add keywords here
keywordstyle=\ttb\color{deepblue},
emph={MyClass,__init__},          % Custom highlighting
emphstyle=\ttb\color{deepred},    % Custom highlighting style
stringstyle=\color{deepgreen},
frame=tb,                         % Any extra options here
showstringspaces=false            % 
}}


% Python environment
\lstnewenvironment{python}[1][]
{
\pythonstyle
\lstset{#1}
}
{}

% Python for external files
\newcommand\pythonexternal[2][]{{
\pythonstyle
\lstinputlisting[#1]{#2}}}

% Python for inline
\newcommand\pythoninline[1]{{\pythonstyle\lstinline!#1!}}


% If you have a file called "university-logo-filename.xxx", where xxx
% is a graphic format that can be processed by latex or pdflatex,
% resp., then you can add a logo as follows:

% \pgfdeclareimage[height=0.5cm]{university-logo}{university-logo-filename}
% \logo{\pgfuseimage{university-logo}}



% Delete this, if you do not want the table of contents to pop up at
% the beginning of each subsection:
\AtBeginSubsection[]
{
\begin{frame}<beamer>
\frametitle{Outline}
\tableofcontents[currentsection,currentsubsection]
\end{frame}
}

% If you wish to uncover everything in a step-wise fashion, uncomment
% the following command:

%\beamerdefaultoverlayspecification{<+->}

\begin{document}


\title{Python快速入门}

%\subtitle{}

% - Use the \inst{?} command only if the authors have different
%   affiliation.
%\author{F.~Author\inst{1} \and S.~Another\inst{2}}
\author{张果}

% - Use the \inst command only if there are several affiliations.
% - Keep it simple, no one is interested in your street address.
\institute[Universities of]
{
厦门大学\quad 王亚南经济研究院}

%\ctexset{today=small} %中文日期
\renewcommand{\today}{\number\year 年 \number\month 月 \number\day 日}
\date{\today}


% This is only inserted into the PDF information catalog. Can be left
% out.
\subject{Talks}




\begin{frame}
\titlepage
\end{frame}

\begin{frame}
\frametitle{目录}
\tableofcontents
% You might wish to add the option [pausesections]
\end{frame}


\section{什么是程序}

\begin{frame}
\frametitle{什么是程序?}
\begin{itemize}
  \item<1-> 程序:输入->输出
  \item<2-> 程序 = 数据结构 + 算法
  \begin{itemize}
    \item<3-> 数据结构:储存、组织数据
    \item<4-> 算法:具体步骤
  \end{itemize}
\end{itemize}
\end{frame}

\begin{frame}
\frametitle{输出}
\begin{python}

>> print 'Hello,world'

Hello,world

\end{python}
\end{frame}


\begin{frame}
\frametitle{输入}
\begin{python}

>> input('please input your favorate number:')

please input your favorate number:7

7

\end{python}
\end{frame}

\section{基本数据类型}
\begin{frame}
\frametitle{基本数据类型}
\begin{itemize}
  \item<1-> 整数: \pythoninline{1}
  \item<2-> 浮点数: \pythoninline{1.1}
  \item<3-> 布尔型: \pythoninline{True},\pythoninline{False}
  \item<4-> 字符串: \pythoninline{'Hello,world'} 
\end{itemize}
\end{frame}

\begin{frame}
\frametitle{基本数据类型}
\begin{itemize}
  \item<1-> 列表: \pythoninline{[1,3,4]}
  \item<2-> 元组: \pythoninline{(1,3,4,5)}
  \item<3-> 集合: \pythoninline{\{1,5,6\}}
  \item<4-> 字典: \pythoninline{\{'xiaoming':1,'xiaohong':2\}}
\end{itemize}
\end{frame}

\section{基本控制语句}
\begin{frame}
\frametitle{if语句}
\begin{python}

>> a = input('please input your number:')

>> if a<10:

..    print 'Hello,world'

\end{python}
\end{frame}

\begin{frame}
\frametitle{for语句}
\begin{python}

>> for i in range(10):

..    print 'Hello,world'

\end{python}
\end{frame}

\begin{frame}
\frametitle{while语句}
\begin{python}

>> a = 0

>> while a<10:

..    print 'Hello,world'

..    a = a + 1

\end{python}
\end{frame}

\section{基本文件操作}

\begin{frame}
\frametitle{写入文件}

\begin{python}

>> import os

>> os.getcwd()  \# get current working dictionary

>> f = open('test.txt','w')

>> f.write('Hello,world')

>> f.close()

\end{python}

\end{frame}

\begin{frame}
\frametitle{读取文件}

\begin{python}

>> import os

>> os.getcwd()

>> f = open('test.txt','r')

>> f.read()

>> f.close()

\end{python}

\end{frame}


\section{函数与模块}

\section{面向对象基础}

% \begin{frame}
% \frametitle{}

% You can create overlays
% \begin{itemize}
%   \item using the \texttt{pause} command:
%   \begin{itemize}
%     \item First item.
%     \pause
%     \item Second item.
%   \end{itemize}
%   \item using overlay specifications:
%   \begin{itemize}
%     \item<3-> First item.
%     \item<4-> Second item.
%   \end{itemize}
%   \item using the general \texttt{uncover} command:
%   \begin{itemize}
%     \uncover<5->{\item First item.}
%     \uncover<6->{\item Second item.}
%   \end{itemize}
% \end{itemize}
% \end{frame}
% 
% \section*{Summary}
% 
% \begin{frame}
% \frametitle<presentation>{Summary}
% 
% \begin{itemize}
%   \item The \alert{first main message} of your talk in one or two lines.
% \end{itemize}
% 
% % The following outlook is optional.
% \vskip0pt plus.5fill
% \begin{itemize}
%   \item Outlook
%   \begin{itemize}
%     \item Something you haven't solved.
%     \item Something else you haven't solved.
%   \end{itemize}
% \end{itemize}
% \end{frame}

\end{document}
